\documentclass[12pt]{extarticle}

\usepackage[T2A]{fontenc}
\usepackage[utf8]{inputenc}
\usepackage[ukrainian]{babel}
\usepackage{amssymb}
\usepackage{amsmath}
\usepackage[dvips]{graphicx}
\usepackage[a4paper,text={19cm,27cm},centering]{geometry}
\usepackage{exercise}
\renewcommand{\ExerciseName}{Задача}

\begin{document}

\begin{Exercise}
За допомогою методу стислих відображень знайти вказане наближення розв’язку інтегрального рівняння Вольтерри. Покласти $\varphi_0=0$. $\varphi(t) = 1 - t\sin t + \int\limits_{0}^{t} s \varphi(s) \mathrm{d}s$, $\varphi_2(t) - ?$
\end{Exercise}

\begin{Exercise}
За допомогою методу стислих відображень знайти вказане наближення розв’язку інтегрального рівняння Вольтерри. Покласти $\varphi_0=0$. $\varphi(t) = \sin t + 0,5\int\limits_{0}^{t} (t+s) \varphi(s) \mathrm{d}s$, $\varphi_2(t) - ?$
\end{Exercise}

\begin{Exercise}
За допомогою методу стислих відображень знайти вказане наближення розв’язку інтегрального рівняння Вольтерри. Покласти $\varphi_0=0$. $\varphi(t) = e^t + \int\limits_{0}^{t} e^{t+2s} \varphi(s) \mathrm{d}s$, $\varphi_2(t) - ?$
\end{Exercise}

\begin{Exercise}
За допомогою методу стислих відображень знайти вказане наближення розв’язку інтегрального рівняння Вольтерри. Покласти $\varphi_0=0$. $\varphi(t) = t + \int\limits_{1}^{t} (t-3s+1)\varphi(s) \mathrm{d}s$, $\varphi_2(t) - ?$
\end{Exercise}

\begin{Exercise}
За допомогою методу стислих відображень знайти вказане наближення розв’язку інтегрального рівняння Вольтерри. Покласти $\varphi_0=0$. $\varphi(t) = 1 + 2t + \int\limits_{0}^{t} (t-s)\varphi(s) \mathrm{d}s$, $\varphi_3(t) - ?$
\end{Exercise}

\begin{Exercise}
За допомогою методу стислих відображень знайти вказане наближення розв’язку інтегрального рівняння Вольтерри. Покласти $\varphi_0=0$. $\varphi(t) = 1 + 3\int\limits_{0}^{t} \varphi(s) \mathrm{d}s$, $\varphi_4(t) - ?$
\end{Exercise}

\begin{Exercise}
За допомогою методу стислих відображень знайти вказане наближення розв’язку інтегрального рівняння Вольтерри. Покласти $\varphi_0=0$. $\varphi(t) = 1 + t^2 - \int\limits_{0}^{t} (t-s+1)^2 \varphi(s) \mathrm{d}s$, $\varphi_2(t) - ?$
\end{Exercise}

\begin{Exercise}
За допомогою методу стислих відображень знайти вказане наближення розв’язку інтегрального рівняння Вольтерри. Покласти $\varphi_0=0$. $\varphi(t) = 1 - 2\sh t + \int\limits_{0}^{t} (t-s+2)\varphi(s) \mathrm{d}s$, $\varphi_2(t) - ?$
\end{Exercise}

\begin{Exercise}
За допомогою методу стислих відображень знайти вказане наближення розв’язку інтегрального рівняння Вольтерри. Покласти $\varphi_0=0$. $\varphi(t) = 1 - \dfrac{1}{2}t^2 + \dfrac{1}{6}\int\limits_{0}^{t} (t-s)^3 \varphi(s) \mathrm{d}s$, $\varphi_2(t) - ?$
\end{Exercise}

\begin{Exercise}
За допомогою методу стислих відображень знайти вказане наближення розв’язку інтегрального рівняння Вольтерри. Покласти $\varphi_0=0$. $\varphi(t) = 3 + t^2 - \int\limits_{0}^{t} (t-s)\varphi(s) \mathrm{d}s$, $\varphi_2(t) - ?$
\end{Exercise}

\begin{Exercise}
За допомогою методу стислих відображень знайти вказане наближення розв’язку інтегрального рівняння Вольтерри. Покласти $\varphi_0=0$. $\varphi(t) = \dfrac{1}{2}t^2 - \int\limits_{0}^{t} (t-s)\varphi(s) \mathrm{d}s$, $\varphi_2(t) - ?$
\end{Exercise}

\begin{Exercise}
За допомогою методу стислих відображень знайти вказане наближення розв’язку інтегрального рівняння Вольтерри. Покласти $\varphi_0=0$. $\varphi(t) = \dfrac{1}{6}t^3 + \int\limits_{0}^{t} (t-s) \varphi(s) \mathrm{d}s$, $\varphi_2(t) - ?$
\end{Exercise}

\begin{Exercise}
За допомогою методу стислих відображень знайти вказане наближення розв’язку інтегрального рівняння Вольтерри. Покласти $\varphi_0=0$. $\varphi(t) = 5 t^2 -3 - \int\limits_{0}^{t} (t-s) \varphi(s) \mathrm{d}s$, $\varphi_3(t) - ?$
\end{Exercise}

\begin{Exercise}
За допомогою методу стислих відображень знайти вказане наближення розв’язку інтегрального рівняння Вольтерри. Покласти $\varphi_0=0$. $\varphi(t) = t \ch t - \int\limits_{0}^{t} s\varphi(s) \mathrm{d}s$, $\varphi_2(t) - ?$
\end{Exercise}

\begin{Exercise}
За допомогою методу стислих відображень знайти вказане наближення розв’язку інтегрального рівняння Вольтерри. Покласти $\varphi_0=0$. $\varphi(t) = t^3 + \int\limits_{0}^{t} (t-s+2) \varphi(s) \mathrm{d}s$, $\varphi_3(t) - ?$
\end{Exercise}

\begin{Exercise}
За допомогою методу стислих відображень знайти вказане наближення розв’язку інтегрального рівняння Вольтерри. Покласти $\varphi_0=0$. $\varphi(t) = t^2 - 1 + \int\limits_{0}^{t} (t-s+1) \varphi(s) \mathrm{d}s$, $\varphi_3(t) - ?$
\end{Exercise}

\begin{Exercise}
За допомогою методу стислих відображень знайти вказане наближення розв’язку інтегрального рівняння Вольтерри. Покласти $\varphi_0=0$. $\varphi(t) = 2 t^2 + 2 + \int\limits_{0}^{t} s \varphi(s) \mathrm{d}s$, $\varphi_3(t) - ?$
\end{Exercise}

\begin{Exercise}
За допомогою методу стислих відображень знайти вказане наближення розв’язку інтегрального рівняння Вольтерри. Покласти $\varphi_0=0$. $\varphi(t) = t - 3 + \int\limits_{0}^{t} (t-s) \varphi(s) \mathrm{d}s$, $\varphi_3(t) - ?$
\end{Exercise}

\begin{Exercise}
За допомогою методу стислих відображень знайти вказане наближення розв’язку інтегрального рівняння Вольтерри. Покласти $\varphi_0=0$. $\varphi(t) = t^3 +2 t^2 + \int\limits_{0}^{t} (t-s+1) \varphi(s) \mathrm{d}s$, $\varphi_2(t) - ?$
\end{Exercise}

\begin{Exercise}
За допомогою методу стислих відображень знайти вказане наближення розв’язку інтегрального рівняння Вольтерри. Покласти $\varphi_0=0$. $\varphi(t) = 3 t^3 + \int\limits_{0}^{t} (t-2s) \varphi(s) \mathrm{d}s$, $\varphi_3(t) - ?$
\end{Exercise}

\begin{Exercise}
За допомогою методу стислих відображень знайти вказане наближення розв’язку інтегрального рівняння Вольтерри. Покласти $\varphi_0=0$. $\varphi(t) = 3t + t^3 + \int\limits_{0}^{t} (2t-s) \varphi(s) \mathrm{d}s$, $\varphi_3(t) - ?$
\end{Exercise}

\begin{Exercise}
За допомогою методу стислих відображень знайти вказане наближення розв’язку інтегрального рівняння Вольтерри. Покласти $\varphi_0=0$. $\varphi(t) = 2t^3 + \int\limits_{0}^{t} (3 t-s) \varphi(s) \mathrm{d}s$, $\varphi_3(t) - ?$
\end{Exercise}

\end{document}
